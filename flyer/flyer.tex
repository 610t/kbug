\documentclass[notumble,combine]{leaflet}
\usepackage[dvipsnames,usenames]{color}
\usepackage{graphicx}
\usepackage{alltt}
\usepackage{url}
\usepackage{ascmac}
\usepackage{comment}
\usepackage{here}
\usepackage{wrapfig}

\makeatletter
% from leaflet.cls
\renewcommand\section{\@startsection{section}{1}{\z@}%
  {-3.5ex \@plus -.75ex}%
  {1ex} %{1.5ex}%
  {\normalfont\large\sectfont\color{NavyBlue}}}
\renewcommand\subsection{\@startsection{subsection}{2}{\z@}%
  {-2.5ex plus -.5ex}%
  {1\p@} %{1ex}%
  {\normalfont\normalsize\sectfont\color{BrickRed}}}
\makeatother

\graphicspath{{figures/}} 

\title{
	\vfill
	\includegraphics[width=\textwidth]{kbug-logo}
	\vfill
	\resizebox{\linewidth}{!}{\bf 関西*BSDユーザ会}%
	\\[\baselineskip]
        \resizebox{\linewidth}{!}{{\bf \url{http://www.kbug.gr.jp/}}}
	\vfill
        \hfill
        \parbox[c]{3cm}{\resizebox{3cm}{!}{\textcolor{red}{\bf{@}}}}
        \hfill
	\parbox[c]{4cm}{\includegraphics[width=4cm]{kyoto_tower}}\\
	\includegraphics[width=\textwidth]{banner}
        \resizebox{\linewidth}{!}{{\bf \url{http://www.ospn.jp/osc2009-kansai/}}}\\
        \resizebox{\linewidth}{!}{2009年7月10日(金),11日(土)}
}

\date{}

\begin{document}
\maketitle
\thispagestyle{empty}
\pagebreak{}
\section{関西*BSDユーザ会ってなに?}
関西 *BSD ユーザ会 (Kansai *BSD Users Group; K*BUG)とは、
BSD系OSのユーザ同士の情報交換のための{\em 場}で、1999年に作
成されました。

年に数度(大体2ヶ月に1度程度)の勉強会や宴会を行っています。

\subsection{K*BSDの基本理念}
K*BUGの基本理念は以下のようになっています
(\url{http://www.kbug.gr.jp/charter.html}より)。

\fbox{\begin{minipage}{\textwidth}
\begin{itemize}
\item 場の提供を目的とする
\item 人のケツは叩くが足は引っ張らない
\item 来るものは拒まず、猿ものは追わず
\item だれでも役員になれる。誰でも役員は止められる
\end{itemize}
\end{minipage}}
\begin{wrapfigure}[8]{r}{3cm}
\begin{center}
 \includegraphics[width=3cm]{CIMG2521}\\
 飲み会は大切
\end{center}
\end{wrapfigure}

少し難しく感じるかもしれないですが、\textcolor{red}{BSDへの愛と情熱}があれば、
あなたが\textcolor{blue}{やりたいと思うことができる場}がK*BUGなのです。

また、メンバーは色々な技術に詳しいですし、技術に関する話も大好きなので、
あなたの疑問などもぶつけてみてください。

\section{最近の活動}
OSC2008 Kansaiからの最近の活動とその内容は以下のとおりです。

もし、興味のあるテーマがあるようでしたら、是非一度遊びにきてください。

\subsection{2009年5月16日(土)@神戸}
\begin{itemize}
\item モダンDNS入門
\item Typolight2.7の紹介
\item llvm/clangでFreeBSD
\item ベーグルボードとInterface付属ボードもってきたよ回覧
\item モバイルギアでNetBSD
\end{itemize}

\subsection{2009年3月7日(土)@大阪}
\begin{itemize}
\item
\end{itemize}

\subsection{2009年1月24日(土)@京都}
\begin{itemize}
\item RemotePad for iPhoneの開発について
\item FreeBSDのsetfibについて
\item FreeBSDでのgainerの利用について
\item Gainer miniとCでの使い方
\item uipaq0ネタ
\item NetBSDのtime\_t 64bit化について
\item TigerのFSEvents APIについて
\end{itemize}

\subsection{2008年12月6日(土)総会@大阪}
\begin{itemize}
\item 
\end{itemize}

\subsection{2008年11月7日(金), 8日(土)KOF2008@大阪}
\begin{wrapfigure}{r}{3cm}
\begin{center}
 \includegraphics[width=3cm]{CIMG3131}%\\
 %OSC2008 Kansaiブースの様子
\end{center}
\end{wrapfigure}

KOF2008\footnote{\url{http://www.ospn.jp/osc2008-kansai/}}で、JNUGさんと共同ブースを設置しました。

\begin{itemize}
\item NetBSDなひととき
\item 展示
\begin{itemize}
\item {\em 恒例、色々なオールドマシンがBSDで動く!!}
\item USL-5P/NetBSDによるLED照明システム
\item NetBSD uvideoで遊ぼう
\item Squeak+Gainer/FreeBSD
\item 趣味の木彫看板
\end{itemize}
\end{itemize}

\subsection{2008年9月6日(土)@大阪}
\begin{itemize}
\item iPod touchを使った何か
\item はじめようNetwork Audio
\item perlでログインプロキシ
\item Typolight WebCMSの概要
\item ぼくのなつやすみ2008(未完)
\item 懇親会: 四季彩
\end{itemize}
\subsection{2008年8月13日(土) OSC2008 Kansai@京都}
\begin{wrapfigure}{r}{3cm}
\begin{center}
 \includegraphics[width=3cm]{CIMG2018}%\\
 %OSC2008 Kansaiブースの様子
\end{center}
\end{wrapfigure}

OSC2008 Kansai\footnote{\url{http://k-of.jp/2008/}}で、JNUGさんと共同ブースを設置しました。

\begin{itemize}
\item NetBSDなひととき
\item 展示
\begin{itemize}
\item BeBox (NetBSD化失敗)
\item NetBSD/O2
\item ネギ降りサーボ + FT232を使ったLCDパネル
\item 魅惑のえびじゅんコレクション
\item Squeak/MGL2@NetBSD/hpcmips
\end{itemize}
\item ブース企画
\begin{itemize}
\item bcbench \footnote{\url{http://www.yagoto-urayama.jp/~oshimaya/nbug/etc/bench/bcbench.html}}チキンレース
\item 夏の京都恒例:電力測定
\end{itemize}
\end{itemize}

\begin{comment}
\subsection{2008年7月13日(土)@京都}
\subsection{2008年5月17日(土)@京都}
\begin{itemize}
\item ZFS
\item NanoBSD紹介
\item ThinkPad X61のSuspend/Resumeについて
\item FT245で遊ぼう
\item ワンセグのお話
\item DebianのOpenSSL関連
\item 懇親会:んまい
\end{itemize}
\begin{center}
 \includegraphics[width=5cm]{CIMG1793}\\
 勉強会での発表: FT245で遊ぼう
\end{center}
\subsection{2008年3月15日(土)@大阪}
\begin{itemize}
\item iPod jailbreak
\item iSCSI関連
\item freebsd-update
\item USB地デジ
\item twitter-API
\item 懇親会 %:????
\end{itemize}

\subsection{2008年2月9日(土)@京都}
\begin{itemize}
\item Life with dtrace
\item サーバのヘッドレス運用について
\item 負け組日記 FreeBSD/amd64
\item kurobox-pro + LCD pro
\item 最近の(BSDでの)Squeak
\item 音声で遊ぼう
\item おみやげ: HD-30 どこでもドライバー
\item 新年会:んまい
\end{itemize}

\begin{center}
 \includegraphics[width=4cm]{CIMG0535}\\
 勉強会での発表: ネギ振りサーボ
\end{center}

\subsection{2007年12月8日(土)@大阪}
\begin{itemize}
\item 総会
\item 忘年会 %:???
\end{itemize}
\subsection{2007年11月10日(土)@KOF2007に合流}
\begin{itemize}
\item JNUGブースに合流%???
\item 懇親会%:???
\end{itemize}
\end{comment}
\subsection{その他}
以下のようなURLで、メンバーの発表が紹介されていますので、活動内容を知るために参考にしてください。
\begin{itemize}
 \item \url{http://hp.vector.co.jp/authors/VA012337/misc/presentation.html}
 \item \url{http://qml.610t.org/FreeBSD/BUG.html}
\end{itemize}
\begin{comment}
\subsection{2007年9月8日(土)@大阪}
\subsection{2007年7月14日(土)@京都}
\subsection{2007年5月19日(土)@大阪}
\subsection{2007年3月17日(土)@京都}
\subsection{2007年1月14日(日)@大阪}
\end{comment}

\pagebreak{}
\section{K*BUG@OSC2009 Kansai}
\subsection{展示}
\begin{itemize}
 \item 魅惑のえびじゅんコレクション「旅するトランク」
 \item Physical Computing @ BSD
 \item \textcolor{red}{\Large 他にもたくさんの○○BSDが…}
\end{itemize}

\section{BSDに関する情報}
\subsection{リンク集}
\begin{itemize}
\item JNUG(Japan NetBSD Users Group)\\
  お隣のブースです\\
  \url{http://www.jp.netbsd.org/}
\item FreeBSD友の会  \url{http://www.jp.freebsd.org/}
\item OpenBSD本家  \url{http://www.openbsd.org/ja/}
\end{itemize}

\pagebreak{}
\section{これからの関連イベント(予定)}
詳細は、Webページをご確認ください。

\begin{itemize}
\item 2009年7月11日(土) 第4回研究会@京都
\item 2009年9月12日(土) 第5回研究会@大阪
\item 2009年11月14日(土) 第6回研究会@京都
\item 2009年12月5日(土) 定期総会+第7回研究会+忘年会@大阪
\end{itemize}

\section{「NetBSDのご紹介」}
以下のように「NetBSDのご紹介」を行いますので、ご参加ください。

\begin{itemize}
\item 日時: 2009/7/10 16:15-17:00
\item 会場: D
\item 講師: 蛯原 純 (The NetBSD Project/株式会社創夢)
\item 主催:日本NetBSDユーザーグループ
%\item \url{http://k-of.jp/2008/list_seminar.html#23}
\end{itemize}

\begin{comment}
BSD系UNIXを取り巻く環境と、将来の展望について議論し、 BSDコミュニティ間の情報交換を行なうBOFセッションです。
4.4BSDの流れをくむFreeBSD/NetBSD/OpenBSDなど、 BSD系UNIXのユーザグループ合同で、BSD系UNIX全般を対象とした幅広いテーマで議論します。
\end{comment}

\vfill

\begin{minipage}{\textwidth}
\begin{boxnote}

\section{kbug-usersメーリングリスト}
K*BUGでは、K*BUGメンバーの情報交換や、イベントなどの情報伝達用に
kbug-usersメーリングリストを用意しています。

K*BUGメンバーは、基本的にはこのメーリングリストを読んでいることが期待
されます。

購読は以下のURLをご参照ください。
\begin{itemize}
 \item \url{http://www.kbug.gr.jp/maillist.html}
\end{itemize}
\end{boxnote}
\end{minipage}

\end{document}  
